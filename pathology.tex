% Template for ICASSP-2013 paper; to be used with:
%          spconf.sty  - ICASSP/ICIP LaTeX style file, and
%          IEEEbib.bst - IEEE bibliography style file.
% --------------------------------------------------------------------------
\documentclass{article}
\usepackage{spconf,amsmath,graphicx,bm}

% Example definitions.
% --------------------
\def\x{{\mathbf x}}
\def\L{{\cal L}}

% Title.
% ------
\title{Paper for pathology review}
%
% Single address.
% ---------------
\name{SAILORS, Shrikanth Narayanan}
\address{Signal Analysis and Interpretation Lab (SAIL),\\
        University of Southern California,  
  Los Angeles, California, USA}  
\begin{document}
\ninept
%
\maketitle
%
\begin{abstract}
 
\end{abstract}
%
\begin{keywords}
Pathological speech, intelligibility 
\end{keywords}
%
\section{Introduction}
\label{sec:intro}

\section{Pathological speech disorders}
%Americal Speech-Language Hearing Association classifies pathological conditions into five categories, viz., (i) speech disorders involving difficulty in producing speech sounds correctly or fluently, (ii) language disorders, when a person has trouble understanding others (ii) social communication  
%as prevention, assesment, diagnosis and treatment of speech, language, social communicative and swallowing disorders in children and adults. 
American Speech-Language Hearing Association \cite{american2008council} identifies pathological disorders into five categories, namely, (i) speech disorders (ii) language disorders (iii) social communication disorders, (iv) cognitive communication disorders and, (v) swallowing disorders. 
Within the characterization of speech pathological disorders, Van der Merwe \cite{van1997characterization,mcneil2009clinical} provides a sound theoretical foundation. 
Citing several speech language pathologists \cite{mcneil1990motoric,kent1987relative,marquardt1984elusive}, she emphasizes on the need for a speech production framework for research and management of pathological disorders. 
To address this, she describes a four level framework characterizing pathological speech as a dysfunction at the levels of linguistic-symbolic planning and speech motor planning. 
Further in \cite{mcneil2009clinical}, Kent, Ballard et al. and Forrest et al. describe assessment/examination methods for motor speech disorders and speech production mechanism. 

\section{Challenges in the domain of understanding pathological conditions}
Theodora and Rahul

\subsection{Defining pathological speech}

\subsection{Subjective Impressions}

\subsection{Speaker variability}

\subsection{Approaches by speech scientists}

\subsubsection{Obtaining data}
Annotations, data demographics, data collection environment

\subsubsection{Features}
Previous studies have attempted to capture the wide variability of pathological speech through various acoustic and phonological features, as well as non-verbal discourse markers.

Voice quality and prosodic features have been extensively used because of their high interpretability and computational efficiency~\cite{van2010computational,tsanas2012novel,bone2014psychologist}, while multi-scale spectro-temporal modulation indices attempt to represent the irregular spectral perturbations of pathological speech~\cite{liss2010discriminating,falk2012characterization,williamson2015automatic}. Vocal source excitation and articulatory features have been proposed in order to capture the malfunctioning of various parts of the speech production system caused by vocal disorders~\cite{falk2012characterization,hahm2015parkinson}. Other efforts have focused on developing distance measures between healthy and pathological speech~\cite{gu2005disordered}. These frame-level features can be incorporated into long-term measures through phone or utterance level functionals~\cite{kim2015automatic}, contour parameterization~\cite{kim2015automatic2}, and other non-linear transformations~\cite{kim2015automatic,an2015automatic,middag2011combining}.

ASR can yield confidence indices of normal speech through lattice posteriors and recognition accuracy metrics~\cite{kim2015automatic,zlotnik2015random,maier2009peaks,sharma2009universal,middag2009automated}. ASR output is further able to provide durational features at the syllable and word level that can be indicative of atypicality~\cite{an2015automatic,duez2006consonant}. Despite the knowledge-driven nature of this approach, challenges of using ASR metrics include the potentially limited vocabulary size, the existence of sparse multilingual data, and the need for speaker-dependent acoustic models.

Non-verbal vocalizations are an essential part of spoken communication for regulating and coordinating discourse. Their atypical occurrence and expression has been related to various neurological and mental disorders~\cite{lake2011listener}. Previous studies have examined the role of fillers, pauses, and laughter in pathological speech~\cite{heeman2010autism,lake2011listener,gupta2014predicting}.

The inherently diverse information present in the speech signal, such as speaker traits, gender and age effects, environmental conditions, etc., makes it hard to disentangle actual pathology-dependent conditions from other factors. Although previous studies have indicated strong correlates of many of the aforementioned features to pathological constructs, careful methodological and experimental planning has to be conducted in order to make sure that the segmentation of the acoustic features space is performed in terms of the relevant pathological effects~\cite{bone2013classifying}. Towards this direction, ecological data capture procedures, reduced-size interpretable features, appropriate statistical analysis, and legitimate experimental validation are encouraged.

\subsubsection{Machine learning}


\section{Case studies}
Jangwon, Naveen, Danny, Jimmy, Bo ....

\subsection{Pathological speech sub-challenge}
Recently, an automatic assessment system for speech intelligiblity and quality has been obtaining lots of attention for assisting speech therapy.
Since manual evaluation by human experts is costly, time-consuming, and subjective, there has been research effort to develop an automatic system to analyze and judge the intelligibility and quality of patients' speech.
The pathology challenge in Interspeech 2012 is the special session where various acoustic features and algorithms were proposed.

The winner of the challege~\cite{kim2012intelligibility} developed multiple expert subsystems which were fused for the final label by using Bayesian fusion models (Naive Bayes or Noise-Majority system).
Individual subsystem focuses on particular aspects of speech, e.g., acoustic similarity to normal speech, prosody, intonation, voice quality and pronunciation quality.
Acoustic features include pitch stylization parameters (quadratic polynomials) for pitch trajectory, speaking rate, harmony-noise ratio, jitter, shimmer, Mel-Frequency Cepstral Coefficients (MFCCs), formants, and phoneme probability feature driven from Automatic Speech Recognition (ASR) lattice.
They employed joint classification scheme: an ad-hoc way of utilizing the high similarity of intelligibility score for the speech audio closely located in the acoustic space.

Ways to handle high feature dimensionality were examined in this challenge.
In order to capture a variety of atypicality in pathological speech, a large number of acoustic features is initially extracted.
Hence, some feature reduction techniques were applied for achieving high accuracy on the test set.
Sparse Gaussian process was introduced in~\cite{lu2012predicting}, where the original features were transformed to a lower dimensional feature space using kernel PCA.
Asymmetric sparse partial least squares regression were also tested in~\cite{huang2012detecting}.
Modified LDA was tested to further reduce the feature dimension from initially reduced dimension by PCA in~\cite{zhou2012automatic}.

\subsection{Parkinson's condition sub-challenge}
Parkinson's disease is one of the most common neurological disorders.
In clinical practice, it is important to track the severity of its symptom.
Using speech signal for monitoring the pregression of Parkinson's disease is an attractive approach, because it is non-invasive, fast, easy-to-obtain and cost-efficient as well as useful for speech treatment.
The task of the Parkinson's condition challenge in Interspeech 2015 was to develop an automatic system to predict the severity of Parkinson disease using a set of speech signals of the patients.

Various features on top of the standard baseline features extracted using openSMILE~\cite{eyben2010themunich} were tested for automatic severity rating.
Rhythmic features adpoted from music information retrieval include beatspectrum~\cite{foote2002audio} and spectral irregularity~\cite{jensen1999timbre}.
The structure of correlations among frame-level speech features were also captured using channel-delay correlation and covariance matrices on the speech waveform, delta-MFCCs and formants, and the articulatory feature streams predicted using the Directions into Velocities of Articulators (DIVA) model~\cite{guenther2006neural,williamson2015segment}.
Using both acoustic and (predicted) articulatory information improved prediction accuracy~\cite{hahm2015parkinson}.

Since the goal was to achieve the best Spearman correlation between predicted and true labels, that is the Unified Parkinson's Disease Rating Scale (UPDRS)~\cite{stebbins1998factor}, participatns tested various regressors, e.g. support vector regressor, random forest, deep neural networks and Gaussian staircase regression model~\cite{williamson2013vocal}. The winning system~\cite{grosz2015assessing} in this challenge used a deep neural network regressor on the average of multiple predictors' scores - the predictors were diversified with different hyperparameter values. 

In the challenge dataset, the UPDRS score was assigned to each speaker, not each utterance.
Hence, judging the final severity score of individual utterances based on all utterances of each speaker (or each UPDRS-score cluster) improved prediction accuracy significantly.
Specifically, incorporating feature selection~\cite{grosz2015assessing} or the information of (predicted) prompt type~\cite{kim2015automatic} into clustering process improved severity rating accuracy as well as clustering accuracy significantly.



\subsection{Depression work}

3. Depression work

4. Other work in ASD, addiction etc. 
Danny, Jimmy, Bo
\vfill\pagebreak

% BiBTeX files (here: strings, refs, manuals). The IEEEbib.bst bibliography
% style file from IEEE produces unsorted bibliography list.
% -------------------------------------------------------------------------
\bibliographystyle{IEEEbib}
\bibliography{strings,refs}

\end{document}
