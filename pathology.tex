% Template for ICASSP-2013 paper; to be used with:
%          spconf.sty  - ICASSP/ICIP LaTeX style file, and
%          IEEEbib.bst - IEEE bibliography style file.
% --------------------------------------------------------------------------
\documentclass{article}
\usepackage{spconf,amsmath,graphicx,bm}

% Example definitions.
% --------------------
\def\x{{\mathbf x}}
\def\L{{\cal L}}

% Title.
% ------
\title{Paper for pathology review}
%
% Single address.
% ---------------
\name{SAILORS, Shrikanth Narayanan}
\address{Signal Analysis and Interpretation Lab (SAIL),\\
        University of Southern California,  
  Los Angeles, California, USA}  
\begin{document}
\ninept
%
\maketitle
%
\begin{abstract}
 
\end{abstract}
%
\begin{keywords}
Pathological speech, intelligibility 
\end{keywords}
%
\section{Introduction}
\label{sec:intro}

\section{Relation to previous work}

\section{Challenges in the domain of understanding pathological conditions}
Theodora and Rahul
\subsection{Defining pathological speech}
\subsection{Subjective Impressions}
\subsection{Speaker variability}
\subsection{Approaches by speech scientists}

\subsubsection{Obtaining data}
Annotations, data demographics, data collection environment

\subsubsection{Features}
Previous studies have attempted to capture the wide variability of pathological speech through various acoustic and phonological features, as well as non-verbal discourse markers.

Voice quality and prosodic features have been extensively used because of their high interpretability and computational efficiency~\cite{van2010computational,tsanas2012novel,bone2014psychologist}, while multi-scale spectro-temporal modulation indices attempt to represent the irregular spectral perturbations of pathological speech~\cite{liss2010discriminating,falk2012characterization,williamson2015automatic}. Vocal source excitation and articulatory features have been proposed in order to capture the malfunctioning of various parts of the speech production system caused by vocal disorders~\cite{falk2012characterization,hahm2015parkinson}. Other efforts have focused on developing distance measures between healthy and pathological speech~\cite{gu2005disordered}. These frame-level features can be incorporated into long-term measures through phone or utterance level functionals~\cite{kim2015automatic}, contour parameterization~\cite{kim2015automatic2}, and other non-linear transformations~\cite{kim2015automatic,an2015automatic,middag2011combining}.

ASR can yield confidence indices of normal speech through lattice posteriors and recognition accuracy metrics~\cite{kim2015automatic,zlotnik2015random,maier2009peaks,sharma2009universal,middag2009automated}. ASR output is further able to provide durational features at the syllable and word level that can be indicative of atypicality~\cite{an2015automatic,duez2006consonant}. Despite the knowledge-driven nature of this approach, challenges of using ASR metrics include the potentially limited vocabulary size, the existence of sparse multilingual data, and the need for speaker-dependent acoustic models.

Non-verbal vocalizations are an essential part of spoken communication for regulating and coordinating discourse. Their atypical occurrence and expression has been related to various neurological and mental disorders~\cite{lake2011listener}. Previous studies have examined the role of fillers, pauses, and laughter in pathological speech~\cite{heeman2010autism,lake2011listener,gupta2014predicting}.

The inherently diverse information present in the speech signal, such as speaker traits, gender and age effects, environmental conditions, etc., makes it hard to disentangle actual pathology-dependent conditions from other factors. Although previous studies have indicated strong correlates of many of the aforementioned features to pathological constructs, careful methodological and experimental planning has to be conducted in order to make sure that the segmentation of the acoustic features space is performed in terms of the relevant pathological effects~\cite{bone2013classifying}. Towards this direction, ecological data capture procedures, reduced-size interpretable features, appropriate statistical analysis, and legitimate experimental validation are encouraged.

\subsubsection{Machine learning}


\section{Case studies}
Jangwon, Naveen, Danny, Jimmy, Bo ....

1. Pathology speech challenge
Jangwon, Naveen

2. Parkinsons challenge
Jangwon, Rahul ..

3. Depression work

4. Other work in ASD, addiction etc. 
Danny, Jimmy, Bo
\vfill\pagebreak

% BiBTeX files (here: strings, refs, manuals). The IEEEbib.bst bibliography
% style file from IEEE produces unsorted bibliography list.
% -------------------------------------------------------------------------
\bibliographystyle{IEEEbib}
\bibliography{strings,refs}

\end{document}
